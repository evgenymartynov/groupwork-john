\section{First Date}

Problem summary: given a date in the Julian calendar, find the number of days that must
be added to determine its corresponding date in the Gregorian calendar.

This problem can be simply achieved by converting the provided date into its respective
'Julian Day Count', which is a uniform count of days from a remote epoch in the past.
At this epoch, -4712 January 1 1200 GMT, the Julian Day Count is considered zero.

To perform conversion from Gregorian date into Julian Day Count we perform the following:

\begin{enumerate}
\item if the month is jan/feb, subtract one from the year and add 12 to the month (they are 
considered to be the 13th and 14th months of the previous year. \\
\item then use the following formula, dropping the fractional part of all multiplications 
  and divisions. \\
\end{enumerate}

\begin{verbatim}
A = Y / 100
B = A / 4
C = 2 - A + B
E = 365.25 * (Y + 4716)
F = 30.6001 * (M+1)
JD = C + D + E + F - 1524.5
\end{verbatim}

NOTE: If we wish to convert a Julian date into its corresponding Julian Day Count we
use this same formula with $C=0$.

This formula determines the Julian Day Number for the beginning of the date represented by
year $Y$, month $M$ and day $D$ and it will always add an additional 12 hours due to the
Julian day beginning at noon GMT. It can be verified by setting Y = 1582, M = 10, D = 14 
to see that it indeed does give you the known Julian Day Count for the beginning of the 
Julian calendar (2299160.5) as represented by its first adjusted day in the Gregorian calendar.

Further conversion from a Julian Day Count to a Gregorian date can be performed using another
formula as follows (where $JD$ is the Julian Day Count):

\begin{verbatim}
Z = JD + 0.5;
F = Z - floor(Z);
W = (Z - 1867216.25) / 36524.25;
X = W / 4;
A = Z + 1 + W - X;
B = A + 1524;
C = (B - 122.1) / 365.25;
D = (365.25 * C;
E = (B - D) / 30.6001;
DAY   = B - D - 30.6001 * E + F;
MONTH = (E > 13) ? E - 13 : E - 1;
YEAR  = (MONTH < 3) ? C - 4715 : C - 4716;
\end{verbatim}

We get our desired answer then by using these formulas to convert the provided Julian
date into its corresponding Gregorian date, using the Julian Day Count as an intermediate
format.

Overall this runs in $O(1)$ time and memory.
