\section{Absurdistan Roads}

Problem summary: given a full $O(n^2)$ matrix of shortest distances in a graph,
work out the original graph. The original is an undirected connected graph with
$n$ vertices and $n$ edges, i.e. a tree with an extra edge.

If we discard the extra edge (assume we remove the largest edge), it can be seen
that the original graph forms an MST for the distance matrix. Justification: any
edge $A-B$ in the matrix that is not part of the original tree will be longer
than any edge in the original tree on a path from $A$ to $B$, since all edge
weights are positive. Thus, we may recover all but one edges from the original
graph by running Prim's in $O(n^2)$.

For the last edge, we are allowed to print out any solution, so we need to get
any edge that works with the given matrix $M$ and our MST. If we generate our
own matrix $M'$ using only the MST edges, it's clear that any differences in $M$
and $M'$ must come from that extra edge being shorter than what the MST allows
for.

Since MST has only $O(n)$ edges, we can compute our $M'$ in $O(n^2)$ by running
DFSs from each node.

If there are no differences then the extra edge in the original graph must be
longer than what we have, and so we may report it as having a huge weight
between any two vertices. This is valid as the MST already generates $M$.

Assuming that there does exist a difference, we may take the smallest number out
of all differing entries in $M$ and $M'$. This suffices since picking a distance
entry that is not the smallest cannot possibly generate the smallest difference.

Provided that our proof of the MST of $M$ being part of the original graph is
correct, this must give a valid solution to the original graph. Full correctness
of this solution is proven by submitting a bug-free code ;-)

Overall this runs in $O(n^2)$ time and memory.
