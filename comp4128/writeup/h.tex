\section{Highway of the Future}

Problem summary: you have a bunch of cars that travel on a highway. Given
departure times ($t_i \leq 10^4$) and velocities ($v_i \leq 100$) of these cars,
what's the highest number of cars that cross each other at the same time? That
is, work out the minimum number of lanes required so that crossing cars can
travel side by side. Times and velocities are integers. The highway length is
100 units, and we are only concerned with crossings that happen on this
interval. There are at most $n \leq 35000$ cars.

Given the above formulation, let us assume that no 2 cars share the same $t_i$
and $v_i$ values (if they do, we simply make a car ``fatter'' in our
calculations). It's clear that each car can cross at most 100 other cars, giving
us a hint on how to approach this.

Work out all crossings --- there are only $O(100n)$ of these, which is feasible
if done in linear time. We... well, kind of use linear time here. Note that we
only care about cars currently on the highway when a car enters the highway
(when computing crossings), and cars differ only by their start time and their
velocities. But, since velocities affect how long the cars would be on a highway
(duh), we get that there are only $$\sum_{v=1}^{100} \frac{100}{v} \approx 100
\log 100 < 500$$ different types of a car that can be on the highway at any
point.

This gives us a $O(500n)$ crossing calculation, which is fast enough. We iterate
over each car in increasing order of start time while maintaining a list of cars
currently on the highway. This list is at most 500 items in length by the above
result and can be unordered, as long as we prune cars that have gone past the
end of the highway.

After that's done, we count the most frequent crossing time (weighted by a car's
``fatness'' from above) and that's our answer.
