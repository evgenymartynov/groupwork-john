\section{Infix to Prefix}

Problem summary: imagine you have an expression composed of numbers (up to 9
digits), unary or binary -, binary +, and brackets. Form an expression tree from
this, where each internal node is an operation and each leaf is a number (i.e.
parse it and drop the brackets). Traverse this in infix order, and concatenate
all node strings to produce a Polish notation expression. What is the smallest
and largest value that this expression can produce?

As an example, ``\texttt{--34}'' can be read as ``\texttt{(- (- 3 4))}'' or as
``\texttt{(- (- 34))}'' or as ``\texttt{(- (- 3) 4)}''.\\

This problem naturally decomposes into DP. Because this is ACM, having an
$O(n^3)$ DP when $n = 1000$ is apparently legit.

There are $O(n^2)$ substrings of the input, and with each operation having an
arity of 1 or 2 means we have $O(n)$ possible ways of splitting a substring into
the operator's arguments. We only have to ensure that each such split is valid.

We then solve the problem by computing $\min$ and $\max$ values for each
substring, defined in terms of each other. The only cases we have to consider
depend on the first character of the substring, and these are:

\begin{enumerate}

  \item A digit: the only option here is that this is a number without any
  operators, so this is valid iff the entire substring has length of at most 9
  and is entirely made up of digits.

  \item A plus: addition has arity 2, so we need to try splitting the substring
  somewhere in the middle. If that's not possible (i.e. all splits produce an
  invalid substring), this substring itself is also invalid.

  \item A minus: this is either unary -, so we drop the first character and
  recurse, or this is the binary - so this is the same case as the plus.

\end{enumerate}

This is trivially $O(n^3)$ with $O(n^2)$ memory, and we did not look for a more
optimised solution.
