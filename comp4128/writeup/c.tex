\section{Card Trick}

Problem summary: you have $n > 10$ cards arranged in a row. These cards can be
assumed to have been chosen randomly with uniform probability. You pick one of
the first 10 cards with uniform probability. You then skip to the next card by
looking at the face value of a card (J, Q, K are 10, A is 11) and jumping this
many cards forward while you can --- if not, you stop. Given a ``trace'' of this
process (including the starting position), but not the rest of the cards, what
is the probability that repeating this process will end on the same card?

First, let us redefine $n$ to be the index of the card we want to stop at (+1
because it's one-indexed). Clearly, this is equivalent to the original
statement.

This is a straightforward DP. If we know a card (it appears in the trace, thus
we know its position and value), we skip to the next state: $$f(i) =
f(i+\textrm{face}(c_i)).$$

If we don't know the card, try all values (with uniform probability): $$f(i) =
\frac{1}{13} \sum_{x \in C} f(i + \textrm{face}(x)),$$ where ``face'' is the
face value as defined above, $c_i$ is the $i$\textsuperscript{th} card, and $C$
is the set of all card faces ($2, 3, \dots, 10, J, Q, K, A$).

The stopping case is when $i > n$, then probability is 0. When $i = n$,
probability is 1 (we're at the goal card).

This runs in $O(n)$ and uses $O(n)$ memory. Writing it bottom-up would use
$O(1)$ memory.

The final answer is spread out over first 10 cards (as per the statement), and
thus is $$\frac{1}{10}\sum_{i=1}^{10} f(i).$$
